\documentclass[12pt,PhD]{Thesis}



% ******** vmargin settings *********
\usepackage{vmargin} %This give you full control over the used page area, it maybe not the idea method in Latex to do so, but I wanted to reduce to amount of white space on the page
\setpapersize{A4}
%\setmargins{3.5cm}%			%linker Rand, left edge
%	  {1cm}%     %oberer Rand, top edge
%           {14.7cm}%		%Textbreite, text width
%           {23.42cm}%   %Texthoehe, text hight
%           {14pt}%			%Kopfzeilenhöhe, header hight
%           {1cm}%   	  %Kopfzeilenabstand, header distance
%           {0pt}%				%Fußzeilenhoehe footer hight
%           {2cm}%    	  %Fusszeilenabstand, footer distance   


%Defining text font profile
\usepackage{t1enc} % as usual
\usepackage[latin1]{inputenc} % as usual
\usepackage{times}	
\usepackage{mathcomp, subfigure}
\usepackage{amsmath}
\usepackage[pdftex]{graphicx}

\pagestyle{plain}
%\renewcommand{\topfraction}{0.99}
%\renewcommand{\bottomfraction}{0.99}
%\renewcommand{\textfraction}{0}


\dept{School of Physics and Astronomy}
\submitdate{2012}

\begin{document}
%Plan
%
\title{Measurements and Simulations of Impedance Reduction Techniques in Particle Accelerators}
%%%%% To be considered? Impedance reduction is still a constant part of all work (MKIs, TCTP, RF fingers, heat loads on ferrites), but not directly applicable as my work for any but MKI.
%%%%% What are the new things presented in the thesis
\author{Hugo Alistair Day}
\principaladviser{Dr. Roger Jones}


\beforeabstract
\prefacesection{Abstract}
A review of the first two years of study are presented. These topics consist of; Simulations of coaxial wire measurements of the impedance of asymmetric devices, coaxial wire measurements of ferrite kicker magnets for use in the SPS and LHC and impedance studies of a number of potential collimator upgrades for the LHC, focusing on the phase 2 secondary collimators for the LHC. Also disucssed is future work towards completion of the PhD and a timetable of writing to ensure timely completion.
\afterabstract
\afterpreface


\tableofcontents

% What is new in the thesis:
% Wire measurements of asymmetric structures
% LMCI with SC and BB impedances - reconstructing Kell-Schnell diagram
% Simulations of wire measurements of structures
% Simulations of large structures (3m of magnets/kicker magnets)
%
%
%

What is new that is presented in thesis?
\begin{itemize}
\item{Localisation of heat loss in trapped modes/damping trapped modes with ferrite}
\begin{itemize}
\item{Localising heat loss using loss density plots in HFSS}
\item{Investigating the effect of ferrite in damping cavity modes - where does the heat loss go and how does R/Q change with the addition of ferrite.}
\end{itemize}
\item{Using the coaxial wire method to measure the quadrupolar impedance of symmetric and asymmetric structures (also the constant transverse term?)}
\item{Simulations of complex devices (is this really that new or exciting?)}
\end{itemize}

Whether this is sufficient, only Crom will tell


\begin{enumerate}
\item{Introduction}
\item{Wakefields, Impedances and the associated Beam Dynamics Effects}
\begin{enumerate}
\item{Theory - Covering the definitions and assumptions of wakefield and impedance studies. Also some simple examples, i.e. resistive wall instability and cavity modes}
\item{Examples of Effects}
\begin{itemize}
\item{Beam-induced heating}
\item{Single bunch and coupled bunch instabilities in the longitudinal and transverse planes - To be introduced and summarised. No in depth analysis needed - others have done it before and better than you can at the moment.}
\item{Example - LMCI with broadband (and space charge - maybe explain implementation) impedances studied in HEADTAIL.}
\item{Comments - Beam induced heating is important to a lot of the work so will be explained in some detail, including reviewing the effects of bunch spectra/bunch length, narrow band, broadband impedances. Also commented on will be location of heating. Will be expanded on further in the examples chapter (TCTP and RF fingers for examples).}
\end{itemize}
\end{enumerate}
\item{Impedance Evaluation Tools (title likely to change)}
\begin{enumerate}
\item{Simulations}
\begin{itemize}
\item{Time Domain}
\item{Frequency Domain - Including eigenmode simulations AND simulations of the coaxial wire technique. Covering symmetric and asymmetric structures}
\end{itemize}
\item{Impedance Measurements (Frequency domain primarily)}
\begin{itemize}
\item{Bench Top}
\begin{itemize}
\item{High Q impedances}
\item{Low Q impedances}
\end{itemize}
\item{Machine Measurements - to demonstrate some knowledge of beam dynamics and because it fits with the section - NOT ALL EQUIPMENT CAN BE MEASURED ON A BENCH}
\begin{itemize}
\item{Longitudinal Measurements (RF phase slippage)}
\item{Transverse Measurements (Tune shift)}
\end{itemize}
\end{itemize}
\end{enumerate}

\item{Impedance Reduction Techniques}
\begin{enumerate}
\item{Resistive Wall Impedance - material conductivity. Using coatings, adjusting material impedance peaks. Phase 2 jaw material as an example}
\item{Gaps - TCTP for geometric, Phase 2 for resistive}
\item{Tapering Angles - TCTP as example (for example the change of tapering angle causing a broadband impedance increase)}
\item{RF fingers - Simulations of RF fingers making the impedance of cavities/transitions}
\item{Damping materials - i.e. the use of ferrite to lower the Q value of resonances. Description of placement and material choice}
\item{Serigraphy - For use on ferrite kicker magnets}
\item{Ceramic chambers - LHC-MKI}
\item{Others}
\end{enumerate}
\item{Indepth studies of 2 Impedance Reduction Techniques}
\begin{enumerate}
\item{LHC-MKI - Particular attention to be paid to the limitations of the technique (electrical breakdown), the measurements and also to a comparison of beam induced heating with different configurations and beam spectrum measurements}
\item{LHC Phase 2 Collimator Design - Secondary collimator and TCTP, comparison to current collimator}
\end{enumerate}
\item{Conclusion}
\end{enumerate}


\end{document}
