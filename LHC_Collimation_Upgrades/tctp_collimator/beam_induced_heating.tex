\subsection{Beam-Induced Heating}
\label{sec:beam-heating-tctp}

As seen in Sec.~\ref{sec:imp-sims-tctp}, the longitudinal impedance of the phase 2 RF design indicates a significant number of beam impedance resonances below 1GHz. Although their contribution in the imaginary component of the beam coupling impedance is not significant enough to be of concern from a stability point of view, the resonances may present a problem from the point of view of beam-induced heating. To fully investigate both the effectiveness of the ferrite in damping the cavity resonances and to identify the locations of the power loss the phase 2 structure is investigated using the frequency domain code HFSS. 

For these simulations we simulate half of the structure (due to the reflective symmetry in the longitudinal plane), using alternatively perfect E-field (enforcing perpendicular electric fields at the boundary) and H-field (enforcing perpendicular magnetic fields at the boundary) boundary conditions at the symmetry plane to indentify and characterise the eigenmodes up to 2GHz (to cover the majority of the beam spectrum). The structure with and without ferrite is simulated to characterise the effect of the ferrite in damping the cavity modes. Simulations are carried out using the following parameters:

\begin{itemize}
\item{Using a 2$^{nd}$ order basis function solver to ensure good resolution of the fields for R/Q and localised loss calculations.}
\item{The ferrite is assumed to be 4A4, materials data is imported from an external data file and interpretted fit is used between data points. An analytical model (see [cite entry] for details).}
\item{We simulate using a single jaw seperation, in this case a half-seperation of 2mm. This is an extremely close jaw seperation, closer in fact than the TCTP collimator would be placed at, but similar to that that the phase 2 collimators would be placed at. This allows some prediction of a worst case scenario for the TCTP and also an analysis of the efficacy of the RF system for the type of operational parameters the phase 2 secondary collimators would be placed at.}
\item{The mesh was auto-generated by the HFSS mesh generator, and run for a convergence criteria of a 0.5\% convergence of the eigenmode frequency between two successive meshes with a 30\% refinement of the mesh between successive solutions.}
\end{itemize}

\begin{figure}
\begin{center}
\includegraphics[width=0.7\textwidth]{LHC_Collimation_Upgrades/figures/longitudinal-impedance-tctp-ferr-freq-dom.pdf}
\end{center}
\label{fig:long-imp-tctp-freq}
\caption{The real component of the longitudinal impedance for the TCTP collimator as simulated by both the time and frequency domains for the case with and without ferrite damping tiles. The strong resonances present in the case without ferrite can be seen to be strongly damped when the ferrite tiles are added. However a substantial broadband component occurs in addition due to the broadened resonance peaks.}
\end{figure}

Here we shall evaluate the resonances as a whole, or a few key resonances from a heating point of view. For a complete listing of the eigenmodes please see App.~\ref{app:tctp-eigenmodes} for a complete breakdown of the TCTP eigenmode simulations. To have a comprehensive review of the heating we consider the following heating possibilities

\begin{itemize}
\item{A beam harmonic occuring exactly on the resonant frequency with a certain bunch profile. Here we consider gaussian and cos$^{2}$ bunch profiles. Parameters for a number of different beam operating modes (summarised in Tab.~\ref{tab:lhc-tctp-heating-para}) are considered.}
\item{Taking theoretical spectra for both 50ns and 25ns bunch spacings. In this case we consider the heating for both nominal operational parameters (1ns bunch length), running conditions from 2012 (bunch length between 1.2-1.4ns) and for HL-LHC parameters. These parameters are summarised in Tab.~\ref{tab:lhc-tctp-heating-para}. Different bunch profiles are considered - gaussian, parabolic and cos$^{2}$ to account for high frequency lobes observed in measured beam spectra.}
\item{Using measured multi-bunch spectra for 50ns bunch spacing measured in the LHC. These measurements are for the beam from injection, through the ramp to squeeze and finally collisions.}
\end{itemize}

\begin{table}
\caption{The LHC operational parameters considered for heating estimates for the TCTP. Operational parameters include the nominal LHC parameters for 25ns bunch spacing, the peak operational intensity for 50ns bunch spacing used in 2012, and the two possible HL-LHC operational schemes, using both 25ns and 50ns bunch spacing. Here the bunch length is assumed to encompass the $4\sigma$ gaussian width.}
\label{tab:lhc-tctp-heating-para}
\begin{center}
\begin{tabular}{c | c | c | c | c }
Operational Mode & $\tau_{b}$ (ns) & $t_{bunch}$ (ns) & $N_{b}$ & $n_{bunches}$ \\ \hline
50ns, 2012 LHC Operation & 1.2 & 50ns & $1.7 \times 10^{11}$ & 1380 \\ \hline
25ns, Nominal LHC Operation & 1.0 & 25ns & $1.15 \times 10^{11}$ & 2808 \\ \hline
HL-LHC 25ns & 1.0 & 25ns & $2.0 \times 10^{11}$ & 2808 \\ \hline
HL-LHC 50ns & 1.0 & 50ns & $3.3 \times 10^{11}$ & 1380 \\ \hline
\end{tabular}
\end{center}
\end{table}

The heating estimates assuming on resonance beam harmonics can be seen in Tab.~\ref{tab:on-res-heating-tctp} for a variety of bunch lengths between 1-1.5ns assuming gaussian and cos$^{2}$ bunch distributions.

\begin{table}
\caption{The power loss of a the TCTP collimator with ferrite for a number of operational modes in the LHC and HL-LHC. All losses are in watts using the parameters found in Tab.~\ref{tab:lhc-tctp-heating-para}}
\begin{center}
\begin{tabular}{c | c | c | c | c | c | c | c | c  }
$\tau_{b}$ (ns) & \multicolumn{2}{| c |}{50ns, 2012} & \multicolumn{2}{| c |}{25ns nominal} & \multicolumn{2}{| c |}{50ns, HL-LHC} & \multicolumn{2}{| c }{25ns, HL-LHC} \\ \hline
 & $P_{loss, g}$ & $P_{loss, c}$ & $P_{loss, g}$ & $P_{loss, c}$ & $P_{loss, g}$ & $P_{loss, c}$ & $P_{loss, g}$ & $P_{loss, c}$ \\ \hline
1.0 & 2.8 & 7.1 & 5.3 & 13.5 & 10.5 & 26.7 & 16.1 & 40.9 \\ \hline
1.1 & 1.9 & 5.1 & 3.6 & 9.8 & 7.2 & 19.4 & 11.0 & 29.5 \\ \hline
1.2 & 1.3 & 3.7 & 2.5 & 7.0 & 5.1 & 13.9 & 7.7 & 21.1 \\ \hline
1.3 & 1.0 & 2.7 &1.8 & 5.0 & 3.6 & 10.0 & 5.5 & 15.2 \\ \hline
1.4 & 0.7 & 1.9 & 1.3 & 3.7 &2.6 & 7.3 & 4.0 & 11.1 \\ \hline
1.5 & 0.5 & 1.4 & 1.0 & 2.7 & 1.9 & 5.4 & 2.9 & 8.2 \\ \hline
\end{tabular}
\end{center}
\end{table}

\subsubsection{Location of Power Deposition}

Locations of power loss - heating on RF fingers and ferrite