\begin{itemize}
\item{Introduction to the concepts behind impedance reduction - we must let the image currents follow a path of miniml resistance}
\item{Commence introduction to different impedance reduction methods}
\end{itemize}
\begin{enumerate}

\section{Tapered and step transition}
\begin{itemize}
\item{Introduction to the changes in impedance with the steepness of a taper}
\end{itemize}

\section{Conductive screening of cavities}
\begin{itemize}
\item{Cavities produce resonant impedances}
\item{We can stop this by making the cavity not visible to the beam by screening it in the frequency range of the beam}
\item{This involves the use of conductive routes for the image current of the beam}
\end{itemize}

\section{Beam screens in kicker magnets}
\begin{itemize}
\item{Kickers are usually constructed from materials with either large losses (magnetic losses - ferrite) or from strong segmentations (laminated kickers) - both bad for impedance}
\item{We need kickers - find a way to screen the beam from kicker material whilst allowing the kicker to operate as intended - capacitively coupled screens etc.}
\end{itemize}

\section{Gaps - covering gaps}
\begin{itemize}
\item{Longitudinal breaks in the beam screen are bad for impedance}
\item{Use some sort of transition - either fixed for non-moveable objects or...}
\item{moveable RF interconnects - like PIMs or other RF fingers constructs}
\end{itemize}

\section{Use of damping materials to de-Q resonant caviities}
\begin{itemize}
\item{Some cavities are unavoidable - for example wire scanners etc.}
\item{We can place a damping material in the cavity to reduce the Q of the cavity - decrease the losses by introducing a lossy material to reduce the Q factor.}
\item{Damping materials can be very thermally sensitive so it is important to know how large the remaining losses are and how much subsequently goes into the damping material}
\end{itemize}

\section{Coating materials near the beam}
\begin{itemize}
\item{}
\end{itemize}

\end{enumerate}