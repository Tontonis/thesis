\section{Time Domain Simulations}



\begin{itemize}
\item{A VERY general introduction to time domain simulations codes. This is not a thesis of computational methods, only one thing you may do with them. Write this way}
\item{General experience - advantages of time domain methods (speed, memory footprint). Weaknesses (mesh resolution of structure, CPU limitation, long simulations times for high-Q resonances)}
\end{itemize}

\subsection{Direct Simulation of a Particle Beam}

A large majority of time domain simulation codes (ECHO, MAFIA, CST Particle Studio, GdfidL) use a method that in effect simulates the passage of a particle beam through the structure and evaluates the subsequent electromagnetic fields in the structure by this signal. This is done in the following way:

\begin{enumerate}
\item{A signal representing the source bunch is defined using a given profile, and then passed through the structure from a defined starting displacement and at a given velocity. Additionally a line of integration is defined along which the witness signal will be taken. This is effect defines a source bunch and a witness particle.}
\item{The simulation is propogated for a given period of time to acquire a significant quantity of the wakefield to correctly analyse the frequency components. It can be seen that for both high-Q resonances and low frequencies this requires a longer wakelength, to encompass the long damping time and correctly resolve the signal this frequency respectively.}
\item{The observed signal is subsequently deconvolved with the source signal to obtain a single particle wakefunction.}
\item{This may then be fourier transformed (using an FFT algorithm or other numerical methods) to acquire the beam coupling impedance.}
\end{enumerate}

These steps are illustrated for clarity in Fig.~\ref{fig:time_domain_beam} using a simple pillbox cavity as an example using the time domain code CST Particle Studio.

\begin{figure}
\subfigure[]{

\label{fig:cst_source_signal}
}
\subfigure[]{

\label{fig:cst_witness_signal}
}
\subfigure[]{

\label{fig:cst_wakepotential}
}
\subfigure[]{

\label{fig:cst_impedance}
}

\caption{An illusration of the \ref{fig:cst_source_signal} source signal and \ref{fig:cst_witness_signal} witness integration in a time domain code. The source signal and the resulting wakefield are shown in \ref{fig:cst_wakepotential}, and the subsequent calculated impedance in \ref{fig:cst_impedance}.}
\ref{fig:time_domain_beam}
\end{figure}

It can readily be seen that by defining either the source signal or witness at different displacements it is possible to acquire both the dipolar and quadrupolar impedances. Additional, by taking the gradient of the transverse impedances any constant transverse impedance terms can also calculated.
